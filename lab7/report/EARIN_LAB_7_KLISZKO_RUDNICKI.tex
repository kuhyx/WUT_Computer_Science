\documentclass{article}[12pt]
\usepackage{listings}
\title{EARIN Lab 7 Report}
\author{Krzysztof Rudnicki, 307585 \and Jakub Kliszko, 303866}
\date{\today}

\lstset{
    frameround=fttt,
    language=Prolog,
    breaklines=true,
    keywordstyle=\bfseries, 
    basicstyle=\ttfamily
    }

\begin{document}
\maketitle
\section{Exercicse Variant 2}
Our task was to write Prolog program which returns number of days between two dates \\ 
Assumptions:
\begin{itemize}
    \item Year is 2023
    \item Number of days is $\leq$ 365
\end{itemize}

Exemplary use:
\begin{lstlisting}
    ?- interval("2205", "0506")
    14

    ?- interval ("0102", "1102")
    10
\end{lstlisting}
Additional assumptions we made based on exemplary use is that we always receive date in "ddmm" format so first we receive 'd' (days) and then 'm' (month) \\ 
If either days or month is a single digit we put '0' in front of the digit to force the string to be 4 characters wide \\ 
Examples: \\
1st of January is represented by "0101" \\ 
2nd of November is represented by "0211" \\ 
15th of January is represented by "1501" \\ 
Another assumption we made is that interval will be always positive, order of dates does not matter so those queries:
\begin{lstlisting}
    ?- interval("2205", "0506")
    14

    ?- interval ("0506", "2205")
    14
\end{lstlisting}
Will give the exact same results. \\ 
Last assumption is that we do not count first date as whole day so:
\begin{lstlisting}
    ?- interval("0101", "0101")
    0

    ?- interval("0101", "3112")
    364
\end{lstlisting}

\section{Program}
Our program successfully performs its task and returns correct number of days based on our assumptions. \\
Program fails and prints out a message "Invalid input" if:
\begin{itemize}
    \item The number of days in the month is too big (for example \lstinline{interval("3205", "0506")})
    \item Month does not exist (for example \lstinline{interval("0113", "0506")})
    \item Input does not make sense (for example \lstinline{interval("xxyy", "0506")})
    \item Day of month is smaller or equal to "00" (for example \lstinline{interval("0011", "0506")})
\end{itemize} 

Program does NOT return error and tries to return correct output (thanks to Prolog magic) for example when given incomplete input like:
\begin{lstlisting}
    ?- interval("106", "0506").
    4
\end{lstlisting}

\subsection{Prolog magic}
Prolog is based on the idea of logic programming. It does not have the concept of functions, it rather operates on predicates and goals. A predicate (a fact or a rule) defines a state of the world and a goal tells Prolog to make that state of the world come true, if possible (in other case it fails).

For example, we can query our program with such command and it will return \emph{true}, because it is a valid state:
\begin{lstlisting}
    ?- month_days('09', 30, 243).
    true
\end{lstlisting}
The following query will fail, because the predicate is not valid for these values:
\begin{lstlisting}
    ?- month_days('10', 28, 123).
    false
\end{lstlisting}
We can however provide Prolog with an unbound values. It will try to find a possible solution and assign them. This is called unification:
\begin{lstlisting}
    ?- month_days('09', DaysInMonth, Days).
    Days = 243,
    DaysInMonth = 30
\end{lstlisting}
If there are multiple possible valid solutions, Prolog will remember them and backtrack to them in case it fails later. User can also ask for another solution if its available by pressing \emph{;}, for instance:
\begin{lstlisting}
    ?- month_days(N, 30, X).
    N = '04',
    X = 90      ;
    N = '06',
    X = 151     ;
    N = '09',
    X = 243     ;
    N = '11',
    X = 304
\end{lstlisting}

\subsection{Modules}
The program consists of three main modules \\ 
\begin{enumerate}
    %\item \lstinline{month_days} which specifies what month string corresponds to what number of days and number of days it takes to reach this month
    %\item \lstinline{day_of_year} which coverts date to number of days in the year used for later calculations, it also checks for erroneous input
    %\item \lstinline{interval} which actually calculates the interval between two dates and returns it
    \item \lstinline{month_days} which defines facts about the months -- how many days each month consists of and how many days it takes to reach this month
    \item \lstinline{day_of_year} which tells how many days have passed since the beginning of the year to a given date
    \item \lstinline{interval} which actually calculates the interval between two dates and writes it
\end{enumerate}

\subsection{Tested examples}
\paragraph{Negative}
\begin{lstlisting}
    % Wrong number of days in month
    ?- interval("3205", "0506")
    Invalid input.
    false.
    
    % non existing month
    ?- interval("0113", "0506").
    Invalid input.
    false.

    % erroneous input
    ?- interval("xxyy", "0506").
    Invalid input.
    false.

    % Number of days equal or smaller than 00
    ?- interval("0011", "0506").
    Invalid input.
    false.

    ?- interval("-1011", "0506").
    Invalid input.
    false.
\end{lstlisting}

\paragraph{Positive}
\begin{lstlisting}
    % Examples from project requirements
    ?- interval("2205", "0506").
    14
    true.

    ?- interval("0102", "1102").
    10
    true.

    % Edge cases, first day of year and last day of year
    ?- interval("0101", "3112").
    364
    true.

    % Edge case, the same date in both inputs
    ?- interval("0101", "0101").
    0
    true.

    % Edge case, input without first digit of day
    ?- interval("2205", "506").
    14
    true.
\end{lstlisting}

\subsection{Differences}
In exemplary use there is no \emph{dot} '.' after query which was necessary for some interpreters (swipl) to actually run the query.\\ 
Also in exemplary use there is no 'true' statement after number of days which Prolog by default displays after successful queries; we decided to not forcefully change it as it would contradict regular Prolog use philosophy.

\section{Challenges}
The biggest challenge was understanding Prolog logic programming and its difference from functional programming paradigm.\\ 
Then it was about finding correct directive for handling characters of date, we decided on \lstinline{atom_chars} and \lstinline{atom_numbers}.\\
Last difficult part was writing the output exactly as in examples using \break\lstinline{write(Interval)} and \lstinline{nl}. 




\end{document}