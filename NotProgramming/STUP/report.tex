\documentclass[12pt]{extreport}


\begin{document}
\begin{center} 
    Krzysztof Rudnicki, 307585, STUP
\end{center}
\begin{center}
    \Huge S K A L A
\end{center}
Była słowem głównym, które wyciągnąłem ze spotkania Sciencepreneurs' Club 
\#8 "Startup a komercjalizacja projektów badawczych". 
Chodzi oczywiście o skalowalność startupów jako \textbf{najważniejszy}
czynnik skłaniający VC do wykupienia lub wsparcia startupu. 
Wydarzenie odbyło się 21 marca 2024 roku o godzinie 16:15 w Centrum Innowacji PW, 
Rektorska 4. Zaproszony pan prof. dr hab. inż. Robert Sitnik 
przedstawiał projekty badawcze, które komercjalizował, 
komercjalizuje i będzie komercjalizował. 
Został przedstawiony produkt skanujący sylwetkę policjanta/policjantki i 
dopasowujący krój munduru w celu zmniejszenia kosztów źle dobranych mundurów 
oraz 3 projekty aktywne:
\begin{enumerate}
\item Smart tracking - do stwierdzania, czy piłka siatkowa wyszła na "aut",
\item Mnemosis - do generowania modeli \textbf{4D} z nagrania aktora,
\item Phibox - do motywowania dzieci do ćwiczeń.
\end{enumerate}

A także projekt Wutif, czyli fundusz wspierający startupy wywodzące się z 
zespołów inżynierskich, szczególnie związanych z Politechniką Warszawską. 
Podobał mi się przede wszystkim luźny klimat spotkania, 
dostosowany pod studentów/studentki, na plus zaproszenie ciekawego gościa, 
osoby, która już coś osiągnęła i ma doświadczenie, którym może się podzielić, 
i długi czas na zadawanie pytań. Z minusów wymieniłbym słaby potencjał networkingowy; 
większość uczestników wydarzenia to studenci, osoby, które dopiero zaczynają 
swoją przygodę ze startupami. Ze spotkania wyniosłem, jak oceniać, czy dana osoba 
przyda się w zespole (zaangażowanie), kiedy sprzedawać rozwiązanie (szybko) i 
poznałem najważniejszą cechę startupów (skalowalność).

\end{document}