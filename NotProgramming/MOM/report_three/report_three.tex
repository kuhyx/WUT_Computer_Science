\documentclass[12pt]{article}
\usepackage[T1]{fontenc}
\usepackage[polish]{babel}
\usepackage[utf8]{inputenc}
\usepackage{hyperref}
\usepackage{listings}
\title{Modelowanie Matematyczne, Projekt 3, Dane nr 1.6}
\author{Krzysztof Rudnicki, 307585}
\setcounter{section}{-1}
\begin{document}
\maketitle 
\section{Wstęp} 
2 Fabryki, F1, F2 \\ 
4 Magazyny, M1, M2, M3, M4 \\ 
6 Klientów, K1, K2, K3, K4, K5, K6 \\ 
\paragraph{Koszty dystrybucji towaru}

\begin{center}
    \begin{tabular}{ | c | c | c | c | c | c | c |  }
        \hline
     zaopatruje & F1 & F2 & M1 & M2 & M3 & M4  \\ 
     \hline
     Magazyny &  &  &  &  &  &   \\ 
     \hline
     M1 & 0.3 & - &  &  &  &   \\ 
     \hline
     M2 & 0.5 & 0.4 &  &  &  &   \\ 
     \hline
     M3 & 1.2 & 0.5 &  &  &  &   \\ 
     \hline
     M4 & 0.8 & 0.3 &  &  &  &   \\ 
     \hline
     Klientów &  &  & &  &  & \\ 
     \hline
     K1 & 1.2 & 1.8 & - & 1.4 & - & -    \\ 
     \hline
     K2 & - & - & 1.2 & 0.3 & 1.3 & -  \\ 
     \hline
     K3 & 1.2 & - & 0.2 & 1.8 & 2.0 & 0.4  \\ 
     \hline
     K4 & 2.0 & - & 1.7 & 1.3 & - & 2.0  \\ 
     \hline
     K5 & - & - & - & 0.5 & 0.3 & 0.5  \\ 
     \hline
     K6 & 1.1 & - & 2.0 & - & 1.4 & 1.5  \\ 
     \hline
    \end{tabular}
    \end{center}
\section{Model Dwukryterialny}
\paragraph{Zbiory}
\begin{itemize}
    \item $i, k \in F$ - Fabryki 
    \item $i, l \in M$ - Magazyny 
    \item $j \in K$ - Klienci
\end{itemize}
\paragraph{Parametery}
\begin{itemize}
    \item $C_{i, j}$ - Koszty transportu dóbr z punktów $i$ (fabryka lub magazyn) do klienta $j$
    \item $C_{k, l}$ - Koszty transportu dóbr z fabryki $i$ do magazynu $k$ 
    \item $P_{i, j}$ - Binarnie określa czy preferencja klienta zostąła spełniona (1) czy nie (0)
    \item $S_j$ - Poziom satysfakcji klienta $j$ - poziom satysfakcji liczymy w zależności od tego ile dany klient zamówił towaru \\ 
    Zadowolenie klientów którzy zamawiają więcej towarów jest traktowane priorytetowo:
    \[ S_1 = \frac{50}{60}  \]
    \[ S_2 = \frac{10}{60}  \]
    \[ S_3 = \frac{40}{60}  \]
    \[ S_4 = \frac{35}{60}  \]
    \[ S_5 = \frac{60}{60}  \]
    \[ S_6 = \frac{20}{60}  \]
    \item $D_j$ - Zapotrzebowanie klienta $j$
\end{itemize}
\paragraph{Zmienne decyzyjne}
\begin{itemize}
    \item $x_{i,j}$ - Liczba dóbr (w tys. ton) przetransportowana z punktu $i$ (fabryka lub magazyn) do klienta $j$
    \item $y_{k, l}$ - Liczba dóbr (w tys. ton) przetransportowana z fabryki $k$ do magazynu $l$
\end{itemize} 
\paragraph{Funkcja celu}
\begin{enumerate}
    \item Minimalizacja kosztów dystrybucji \\ 
    \[ Min(\sum_{i, j} C_{i, j} * x_{i, j} + \sum{i, k} C_{i, k} * y_{i, k}) \]
    \item Maksymalizacja satysfakcja klienta \\ 
    W celu maksymalizacji satysfakcji klienta policzymy ile z dostaw do klientów odbyło się z preferencyjnych źródeł \\
    \[ Max(\sum_{j} S_j * P_{i,j} * x_{i, j}) \] 
\end{enumerate}
\paragraph{Funkcja celu alpha beta}
\[ \alpha * (Min(\sum_{i, j} C_{i, j} * x_{i, j} + \sum{i, k} C_{i, k} * y_{i, k})) + \beta * (Max(\sum_{j} S_j * P_{i,j} * x_{i, j}))  \]
\paragraph{Ograniczenia}
Miesięczne możliwości produkcyjne fabryk
\begin{equation}
    \sum_j x_{F1, j} + \sum_k y_{F1, k} \leq 150
\end{equation}
\begin{equation}
    \sum_j x_{F2, j} + \sum_k y_{F2, k} \leq 200
\end{equation}
Miesięczna ilośc obsługiwanego towaru przez magazyny 
\begin{equation}
    \sum_j x_{M1, j} \leq 70
\end{equation}
\begin{equation}
    \sum_j x_{M2, j} \leq 50
\end{equation}
\begin{equation}
    \sum_j x_{M3, j} \leq 100
\end{equation}
\begin{equation}
    \sum_j x_{M4, j} \leq 40
\end{equation}
Spełnienie preferencji klienta 
\begin{equation}
    \sum_i x_{i, j} = D_j
\end{equation}
Wartości niezerowe 
\begin{equation}
    x_{i, j}, y_{i, k} \geq 0
\end{equation}

\section{Implementacja}
Do implementacji użyty został python z biblioteką pulp \href{https://coin-or.github.io/pulp/index.html}{https://coin-or.github.io/pulp/index.html} \\ 
Dzięki temu wykorzystujemy zarówno łatwość pythona jak i możliwości używania różnych solverów (CBC, GLPK, CPLEX, Gurobi...) przez pulpa \\ 
\begin{lstlisting}[language=Python, caption={Import bilbioteki pulp i bibliotek używanych do wykresu}]
    import pulp
    import sys
\end{lstlisting}
\begin{lstlisting}[language=Python, caption={Iniicjalizacja modelu}]
    model = pulp.LpProblem("Optimal_Distribution", pulp.LpMinimize)
\end{lstlisting}

\begin{lstlisting}[language=Python, caption={Zdefiniowanie zmiennych i parametrów}]
    fabryki = ['F1', 'F2']
    magazyny = ['M1', 'M2', 'M3', 'M4']
    dostawcy = fabryki + magazyny 
    
    klienci = ['K1', 'K2', 'K3', 'K4', 'K5', 'K6']
    
    koszt = {
        'F1': {'M1': 0.3, 'M2': 0.5, 'M3': 1.2, 'M4': 0.8,
        'K1': 1.2, 'K2': 999, K3': 1.2, 'K4': 2.0,
        'K5': 999, 'K6': 1.1},
        'F2': {'M1': 999, 'M2': 0.4, 'M3': 0.5, 'M4': 0.3,
        'K1': 1.8, 'K2': 999, 'K3': 999, 'K4': 999,
        'K5': 999, 'K6': 999},
        'M1': {'K1': 999, 'K2': 1.2, 'K3': 0.2, 'K4': 1.7,
        'K5': 999, 'K6': 2.0},
        'M2': {'K1': 1.4, 'K2': 0.3, 'K3': 1.8, 'K4': 1.3,
        'K5': 0.5, 'K6': 999},
        'M3': {'K1': 999, 'K2': 1.3, 'K3': 2.0, 'K4': 999,
        'K5': 0.3, 'K6': 1.4},
        'M4': {'K1': 999, 'K2': 999, 'K3': 0.4, 'K4': 2.0,
        'K5': 0.5, 'K6': 1.6}
    }
    P = pulp.LpVariable.dicts(
        "P", [(i, j) for i in dostawcy for j in klienci],
        cat='Binary')
    maksymalne_zamowienie = 60
    poziom_satysfakcji = {
        'K1': 50 / maksymalne_zamowienie,
        'K2': 10 / maksymalne_zamowienie,
        'K3': 40 / maksymalne_zamowienie,
        'K4': 35 / maksymalne_zamowienie,
        'K5': 60 / maksymalne_zamowienie,
        'K6': 20 / maksymalne_zamowienie}
    preferencja_klienta = {
        'K1': ['F2'], 'K2': ['M1'], 'K3': ['M2', 'M3'],
        'K4': ['F1'], 'K5': [], 'K6': ['M3', 'M4']}
    suma_satysfakcji = pulp.lpSum(
        [poziom_satysfakcji[j] * P[(i, j)]
        for i in dostawcy for j in klienci])
\end{lstlisting}

\begin{lstlisting}[language=Python, caption={Zmienne decyzyjne}]
    x = pulp.LpVariable.dicts(
        "x",
        [(i, j) for i in dostawcy for j in klienci],
        lowBound=0, cat='Integer')
    y = pulp.LpVariable.dicts("y",
    [(i, k) for i in fabryki for k in magazyny],
    lowBound=0, cat='Integer')
\end{lstlisting}

\begin{lstlisting}[language=Python, caption={Funkcje celu}]
    koszt_dystrybucji = pulp.lpSum(
        [koszt[i][j] * x[(i, j)]
        for i in dostawcy for j in klienci])
    koszt_magazynowania = pulp.lpSum(
        [koszt[i][k] * y[(i, k)]
        for i in fabryki for k in magazyny])
    alpha = 0.5
    beta = 0.5
    model += alpha
    * (koszt_dystrybucji + koszt_magazynowania)
    - beta * suma_satysfakcji
\end{lstlisting}

\begin{lstlisting}[language=Python, caption={Ograniczenia}]
for i in fabryki:
    model += pulp.lpSum(
        [x[(i, j)] for j in klienci]
        + [y[(i, k)] for k in magazyny])
        <= mozliwosci_fabryki[i]

for k in magazyny:
    model += pulp.lpSum(
        [x[(k, j)] for j in klienci]
        ) <= pojemnosc_magazynu[k]

for j in klienci:
    model += pulp.lpSum(
        [x[(i, j)] for i in dostawcy]) == zamowienia_klientow[j]
\end{lstlisting}

\begin{lstlisting}[language=Python, caption={Rozwiązanie problemu}]
    model.solve()
\end{lstlisting}
\

\begin{lstlisting}[language=Python, caption={Przedstawienie wyników}]
for v in model.variables():
    print(v.name, "=", v.varValue)
\end{lstlisting}

\section{Rozwiązanie efektywne}
Aby zdefiniować rozwiązanie efektywne sprawdzamy sumaryczy obu funkcji celu dla wartości $\alpha$ i $beta$ od 1 do 10 \\ 
w tym celu napisany został kod który modyfikuje wartości $\alpha$ i $\beta$ w pętli 

\begin{lstlisting}[language=Python, caption={Wyznaczanie alpha i beta}]
    koszt_wyniki = []
    zadowolenie_wyniki = []
    wyniki_funkcji = []
    maksymalny_wynik = 0;
    for alpha in range(0, 11):
        beta = 10 - alpha + sys.float_info.epsilon
        alpha /= 10.0
        beta /= 10.0
        # Update objective function
        model.objective = alpha * koszt_dystrybucji
        - beta * suma_satysfakcji
    
        # Solve the model
        model.solve()
        print(alpha)
    
        # Record the wyniki
        calkowity_koszt = pulp.value(koszt_dystrybucji)
        calkowite_zadowolenie = pulp.value(suma_satysfakcji)
        wynik_funkcji = pulp.value(alpha * koszt_dystrybucji
        - beta * suma_satysfakcji)
        koszt_wyniki.append(calkowity_koszt)
        zadowolenie_wyniki.append(calkowite_zadowolenie)
        if wynik_funkcji > maksymalny_wynik:
            maksymalny_wynik = wynik_funkcji
        wyniki_funkcji.append(wynik_funkcji)
        
    print("maksymalny_wynik", maksymalny_wynik, wyniki_funkcji)

\end{lstlisting}
Najwyższy wynik zostął uzyskany dla $\alpha = 10$ i $\beta = 0$ i wynosił on \textbf{156.5}

\paragraph{Symulacja procesu podejmowania decyzji}
Przeprowadzone zostały symulacje dla 5 sytuacji:
\begin{enumerate}
    \item Tylko minimalizacja kosztów $\alpha = 1.0$ $\beta = 0.0$
    \item Priorytet na minimalizacji kosztów $\alpha = 0.8$ $\beta = 0.2$
    \item Równy podział $\alpha = 0.5$ $\beta = 0.5$
    \item Priorytet na satysfakcji klientów $\alpha = 0.2$ $\beta = 0.8$
    \item Tylko satysfakcja klientów $\alpha = 0.0$ $\beta = 1.0$
\end{enumerate}

Ponownie w celu przeprowadzenia symulacji napisano kod w pythonie 
\begin{lstlisting}[language=Python]
# Pseudo-code, assuming model setup as previously discussed
scemariusze = [
    (1.0, sys.float_info.epsilon),
    (0.8, 0.2),
    (0.5, 0.5),
    (0.2, 0.8),
    (sys.float_info.epsilon, 1.0)]
wyniki = []

for alpha, beta in scemariusze:
    model.objective = alpha * koszt_dystrybucji
        - beta * suma_satysfakcji

    model.solve()

    calkowity_koszt = pulp.value(koszt_dystrybucji)
    calkowite_zadowolenie = pulp.value(suma_satysfakcji)
    wyniki.append(
        (alpha, beta, calkowity_koszt, calkowite_zadowolenie)
        )

for idx, (alpha, beta, koszt, zadowolenie) in enumerate(wyniki):
    print(f"Krok {idx+1}:")
    print(f"  koszt: {alpha}, zadowolenie: {beta}")
    print(f" 
     Calkowity koszt: {koszt}, 
     zadowolenie klienta: {zadowolenie}\n")
\end{lstlisting}

    \end{document}