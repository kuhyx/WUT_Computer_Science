\documentclass[12pt]{article}
\usepackage[utf8]{inputenc}
\usepackage{amsmath}
\usepackage{graphicx}
\usepackage[utf8]{inputenc}
\usepackage[T1]{fontenc}
\usepackage{natbib}
\usepackage{soul}

\title{Analiza statystyczna przyznawania funduszy UE gminom}
\author{Krzysztof Rudnicki, Michał Sar}
\date{\today}

\begin{document}

\maketitle

\tableofcontents

\begin{abstract}
    This is a brief summary of your study, its results, and major conclusions.
\end{abstract}


\section{Wstęp}
\subsection{Kontekst}
W 2024 mija 20 lat od wstąpienia Polski do Unii Europejskiej \cite{1}. Od tamtej pory bilans Polski w stosunku do Brukseli wynosi 175 miliardów euro na plus dla Polski \cite{2} W samym 2023 roku Polska otrzymała z UE prawie 3.5 miliarda złotych, wpłacająć niecały miliard złotych \cite{3} W naszej pracy ponawiamy analizę statystyczną wykonaną sprzed 7 lat, na nowych danych, od początku roku 2014 do końca roku 2023
\subsection{Cel}
Celem pracy jest sprawdzenie jakie dane na temat gminy najbardziej korelują z liczbą przyznanych funduszy Unii Europejskiej danej gminy
\subsection{Hipoteza} 
\ul{Gęstość zaludnienia jest \textbf{najważniejszym} czynnikiem wpływającym na przyznanie środków unijnych} \\
\subsection{Metoda badawcza}
\begin{enumerate}
    \item Zebrać dane UE
    \item Zebrać dane gmin
    \item Połączyć dane po numerze TERYT
    \item Przeanalizować dane 
    \item Wyświetlić wyniki
\end{enumerate}
\subsection{Wyniki}

\section{Omówienie rozdziałów}
Na początku artykułu przedstawiamy czemu wybraliśmy taki temat, co chcemy osiągnąć naszą pracą, w jaki sposób chcemy to osiągnąć i jaki rezultat ostatecznie udało nam się pokazać \\  
Następnie opisujemy istniejącą literaturę na temat środków Unijnych z którą się zapoznaliśmy i przedstawiamy w czym różni się nasza praca od istniejących \\ 
Potem tłumaczymy nasz proces badawczy, w jaki sposób zbieraliśmy i łączyliśmy dane, jak je analizowaliśmy i jak przedstawialiśmy wyniki \\ 
Kontynując, pokazujemy co otrzymaliśmy ostatecznie w wyniku naszej pracy \\ 
Przedostatni rozdział zajmuje się dyskusją wyników, przedstawiamy co udało nam się osiągnać i dlaczego, czego nie udało nam się osiągnąć i dlaczego oraz przede wszystkim konfrontujemy wynik z naszą hipotezą \\ 
Na końcu podsumowujemy całą pracę i przedstawiamy spis literatury z której korzystaliśmy

\section{Opis literatury}
\subsection{Decision trees: from efficient prediction to responsible AI \cite{4}}
Artykuł poświęcony jest omówieniu drzew decyzyjnych, rozpoczyna od zdefiniowania czym drzewo decyzyjne jest, jakie są jego unikalne cechy, gdzie jest stosowane, jakie ma wady i potencjalne zagrożenia oraz jak można je zminimalizować \\ 
Wybraliśmy ten artykuł gdyż opisuje jedną z głównych metod którą zamierzamy stosować w naszym procesise badawczym do przeanalizowania danych 
\subsection{Application of Successful EU Funds Absorption Models to Sustainable Regional Development \cite{5}}
Artykuł wykorzystał ankiety pytając ankietowanych o to jak efektywnie wykorzystywane były fundusze UE w Polsce, Słowenji, Węgrzech i Chorwacji pod względe, 
\section{Proces badawczy}
\subsection{Zbieranie danych}
\subsection{Analiza danych}
\subsection{Przedstawienie wyników}

\section{Wyniki}


\section{Dyskusja}


\section{Konkluzja}


\bibliographystyle{plain}
\bibliography{references}

\end{document}
